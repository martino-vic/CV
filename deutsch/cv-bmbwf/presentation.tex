%-------------------------------------------------------------------------------
%	SECTION TITLE
%-------------------------------------------------------------------------------
\cvsection{Präsentationen}


%-------------------------------------------------------------------------------
%	CONTENT
%-------------------------------------------------------------------------------
\vspace{-0.3cm}
\begin{small}\textit{* als Vortragender}\end{small}

%-------------------------------------------------------------------------------
\cvsubsection{Eingeladene Vorträge}
%-------------------------------------------------------------------------------

\begin{cvpubs}
    \cvpub {6. September 2023. \textit{Inferring language contact through rule-based computational borrowing detection}. Interdisziplinäre Seminarreihe zur menschlichen Diversität, Turku, Finnland.}
\end{cvpubs}

%-------------------------------------------------------------------------------
\cvsubsection{Beigetragene Präsentationen}
%-------------------------------------------------------------------------------

\begin{cvpubs}

    \cvpub{\textbf{Viktor Martinovi{\'c}}, Tihomir Rangelov*. 12. Juli 2023. \textit{Automatic detection of phonological adjustments in Bislama’s lexicon of English origin}. Vanuatu Languages Conference, Port Vila, Vanuatu.}
\end{cvpubs}

\cvsubsection{Teilnahme}

\begin{cvpubs}
    \cvpub {23.–28. Juli 2019. 10th Global WordNet Conference, Wroclaw, Polen}
\end{cvpubs}
%-------------------------------------------------------------------------------
\cvsubsection{Konferenzpräsentationen}

\begin{cvpubs}
    \cvpub {21.–26. August 2022, \textit{Gothic loanwords in Hungarian? A case study of computer-assisted borrowing detection}. Congressus XIII Internationalis Fenno-Ugristarum, Wien, Österreich}
\end{cvpubs}

\begin{cvpubs}
    \cvpub {12.–20. November 2021, \textit{Workshop: \textipa{f@nEtIk tr\ae nskraIb@rz In "paIT@n}. From orthography to IPA}. 70. Studentische Tagung Sprachwissenschaft, Wien, Österreich}
\end{cvpubs}

\begin{cvpubs}
    \cvpub {23.–24. September 2021. \textit{Rule-based vs. machine learning models in automated borrowing detection}. The Seventh International Workshop on
Computational Linguistics of Uralic Languages, Syktyvkar, Russland}
\end{cvpubs}

\begin{cvpubs}
    \cvpub {6.–8. September 2021. \textit{loanpy – a framework for computer-aided borrowing detection}. Protolang 7, Düsseldorf, Deutschland}
\end{cvpubs}

\begin{cvpubs}
    \cvpub {23.–26. Februar 2021. \textit{Loanpy: A framework for computer-aided borrowing detection}. 43. Jahrestagung der Deutschen Gesellschaft für Sprachwissenschaft, Freiburg, Deutschland}
\end{cvpubs}

\begin{cvpubs}
    \cvpub {19.–22. November 2020. \textit{Towards a computer-aided framework for borrowing detection: Gothic loanwords in Hungarian?} 68. StuTS, Berlin, Deutschland}
\end{cvpubs}

\begin{cvpubs}
    \cvpub {17.–19. August 2020, \textit{Gothic-Hungarian lexical contacts}. Pre-Congress of Congressus XIII Internationalis Fenno-Ugristarum, Wien, Österreich}
\end{cvpubs}

\begin{cvpubs}
    \cvpub {1. Juli 2019, \textit{Gothic loanwords in Hungarian? Computational approaches to borrowing detection}. Fakultätsöffentliche Präsentation beim Dies Doctoralis der Philologisch-Kulturwissenschaftlichen Fakultät der Universität Wien, Wien, Österreich}
\end{cvpubs}

\begin{cvpubs}
    \cvpub {6.–8. Juni 2019, \textit{Hungaro-Gothica: Are there Gothic loanwords in Hungarian?} 3rd Budapest Linguistics Conference, Budapest, Ungarn}
\end{cvpubs}

\begin{cvpubs}
    \cvpub {9.–10. Mai 2019, \textit{Hungarian-Gothic contact?} Workshop der Sprachwissenschaftlichen DoktorandInnen der Wiener Germanistik and friends, Wien, Österreich}
\end{cvpubs}

%\begin{cvpubs}
%    \cvpub {PhD seminar uni helsinki may 2020}
%\end{cvpubs}

%-------------------------------------------------------------------------------

