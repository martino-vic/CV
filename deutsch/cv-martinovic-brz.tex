%!TEX TS-program = xelatex
%!TEX encoding = UTF-8 Unicode
% Awesome CV LaTeX Template for CV/Resume
%
% This template has been downloaded from:
% https://github.com/posquit0/Awesome-CV
%
% Author:
% Claud D. Park <posquit0.bj@gmail.com>
% http://www.posquit0.com
%
% Template license:
% CC BY-SA 4.0 (https://creativecommons.org/licenses/by-sa/4.0/)
%


%-------------------------------------------------------------------------------
% CONFIGURATIONS
%-------------------------------------------------------------------------------
% A4 paper size by default, use 'letterpaper' for US letter
\documentclass[11pt, a4paper]{awesome-cv}
\usepackage{graphicx}  % for photo!
 
% Configure page margins with geometry
\geometry{left=1.4cm, top=.8cm, right=1.4cm, bottom=1.8cm, footskip=.5cm}

% Color for highlights
% Awesome Colors: awesome-emerald, awesome-skyblue, awesome-red, awesome-pink, awesome-orange
%                 awesome-nephritis, awesome-concrete, awesome-darknight
\colorlet{awesome}{awesome-red}
% Uncomment if you would like to specify your own color
% \definecolor{awesome}{HTML}{CA63A8}

% Colors for text
% Uncomment if you would like to specify your own color
% \definecolor{darktext}{HTML}{414141}
% \definecolor{text}{HTML}{333333}
% \definecolor{graytext}{HTML}{5D5D5D}
% \definecolor{lighttext}{HTML}{999999}
% \definecolor{sectiondivider}{HTML}{5D5D5D}

% Set false if you don't want to highlight section with awesome color
\setbool{acvSectionColorHighlight}{true}

% If you would like to change the social information separator from a pipe (|) to something else
\renewcommand{\acvHeaderSocialSep}{\quad\textbar\quad}


%-------------------------------------------------------------------------------
%	PERSONAL INFORMATION
%	Comment any of the lines below if they are not required
%-------------------------------------------------------------------------------
% Available options: circle|rectangle,edge/noedge,left/right
\photo{profile.jpg}
\name{Viktor}{Martinović, MA}
\position{Computerlinguist | Datenanalyst | Internationale Beziehungen | Doktorand}
\address{Karmeliterhofgasse 7-9/3/10, 1150 Wien, Österreich}

\mobile{+43 681 20321322}
\email{vik.martinovic@gmail.com}
\dateofbirth{2. April 1995}
%\homepage{www.posquit0.com}
\github{martino-vic}
\linkedin{martino-vic}
% \gitlab{gitlab-id}
% \stackoverflow{SO-id}{SO-name}
% \twitter{@twit}
% \skype{skype-id}
% \reddit{reddit-id}
% \medium{medium-id}
% \kaggle{kaggle-id}
% \hackerrank{hackerrank-id}
% \googlescholar{-oJetPMAAAAJ&hl=de&oi=ao}{scholar}
%% \firstname and \lastname will be used
% \googlescholar{googlescholar-id}{}
% \extrainfo{extra information}

\quote{``Motivierter und technisch versierter PhD-Kandidat mit Hintergrund in Computerlinguistik, Datenmanagement und Softwareentwicklung, der sein Fachwissen am Bundesrechenzentrum Österreich einbringen möchte. Erfahrung in der Implementierung und Optimierung datengesteuerter Lösungen, mit Kenntnissen in Python, SQL und Cloud-basierten Technologien. Versiert in Prozessautomatisierung, Datenverwaltung und der Arbeit in funktionsübergreifenden Teams. Ich möchte meine analytischen und Programmierkenntnisse in einem dynamischen Umfeld einsetzen, das die digitale Transformation öffentlicher Dienste unterstützt. Engagiert in kontinuierlichem Lernen und der Mitarbeit an innovativen IT-Lösungen für den öffentlichen Sektor. Fließend in Deutsch und Englisch."}


%-------------------------------------------------------------------------------
\begin{document}

% Print the header with above personal information
% Give optional argument to change alignment(C: center, L: left, R: right)
\makecvheader

% Print the footer with 3 arguments(<left>, <center>, <right>)
% Leave any of these blank if they are not needed
\makecvfooter
  {\today}
  {Viktor Martinović~~~·~~~Curriculum Vitae}
  {\thepage}


%-------------------------------------------------------------------------------
%	CV/RESUME CONTENT
%	Each section is imported separately, open each file in turn to modify content
%-------------------------------------------------------------------------------
\cvsection{Beschäftigung}
\begin{cventries}

  \cventry
    {Doktorand} % Degree
    {Max-Planck-Institut für evolutionäre Anthropologie} % Institution
    {Leipzig} % Location
    {Oktober 2022 - September 2023} % Date(s)
    {
      \begin{cvitems} % Description(s) bullet points
      \item{Implementierung, Governance, Warehousing und Verwaltung von Forschungsdaten im \underline{\href{https://cldf.clld.org/}{CLDF-Standard}}.}
      \item{Entwicklung meiner eigenen Open-Source-Python-Bibliothek \underline{\href{https://pypi.org/project/loanpy/}{LoanPy}} für quantitative Analysen.}
      \end{cvitems}
    }
    
  \cventry
    {Verwaltungspraktikant} % Degree
    {Österreichisches Außenministerium, Abt. für Südosteuropa} % Institution
    {Wien} % Location
    {Juli - Dezember 2019} % Date(s)
    {
      \begin{cvitems} % Description(s) bullet points
        \item{Sammeln und Aufbereiten von Informationspaketen und Briefing Notes.}
      \end{cvitems}
    }

  \cventry
    {Wissenschaftliche Hilfskraft} % Degree
    {Universität Wien} % Institution
    {Wien} % Location
    {Oktober 2017 - Januar 2018} % Date(s)
    {
    \begin{cvitems}
    \item{Projektmanagement zur Organisation von Konferenzdinnern mit vier Botschaften.}
    \end{cvitems}
    }
    
%  \cventry
 %   {Praktikant} % Degree
  %  {Ungarische Botschaft Paris} % Institution
   % {Paris} % Location
    %{Juni - August 2016} % Date(s)
    %{
%    \begin{cvitems}
%    \item{Analyse und Übersetzung von täglichen Ereignissen.}
%    \end{cvitems}
%    }

%  \cventry
 %   {Präsenzdiener} % Degree
  %  {Österreichisches Bundesheer} % Institution
   % {Horn \& Wien} % Location
  %  {März - August 2014} % Date(s)
   % {
%    \begin{cvitems}
 %   \item{1. Gardekompanie}
  %  \end{cvitems}
    %}
        
\end{cventries}

%-------------------------------------------------------------------------------
%	SECTION TITLE
%-------------------------------------------------------------------------------
\cvsection{Education}


%-------------------------------------------------------------------------------
%	CONTENT
%-------------------------------------------------------------------------------
\begin{cventries}

%---------------------------------------------------------
  \cventry
    {University of Vienna, Finno-Ugric Studies \& Computer Science} % Degree
    {PhD in Computational Evolutionary Linguistics} % Institution
    {Vienna} % Location
    {October 2018 - ongoing} % Date(s)
    {
      \begin{cvitems} % Description(s) bullet points
        \item{Thesis: Process automation in loanword research.}
        \item{Training, evaluation, and comparison of LLMs, RNNs, and rule-based predictive models.}
        \item{Research stays in Helsinki (2019/20), Reykjavík (2021/22), and Leipzig (2022/23).}
      \end{cvitems}
    }

  \cventry
    {University of Vienna, Finno-Ugric Studies} % Degree
    {MA in Finno-Ugric Studies (Ø 1.5)} % Institution
    {Vienna} % Location
    {January 2017 - June 2018} % Date(s)
    {
      \begin{cvitems} % Description(s) bullet points
        \item{Thesis on the manual identification of Germanic loanwords in Hungarian.}
      \end{cvitems}
    }

  \cventry
    {Vienna School of International Studies | Diplomatische Akademie} % Degree
    {Diploma in International Relations} % Institution
    {Vienna} % Location
    {October 2017 - June 2018} % Date(s)
    {
    \begin{cvitems} % Description(s) bullet points
        \item{International and EU Law, Micro- and Macroeconomics, History, Politics.}
      \end{cvitems}
    }
    
  \cventry
    {University of Vienna \& INALCO (Institut National des Langues et Civilisations Orientales)} % Degree
    {BA in Hungarian Studies (Ø 1.6)} % Institution
    {Vienna \& Paris} % Location
    {2014 - 2016 - 2017} % Date(s)
    {
      \begin{cvitems} % Description(s) bullet points
        \item{Thesis on the history of Germanic loanwords in Hungarian.}
      \end{cvitems}
    }
        
  \cventry
    {Schottengymnasium \& Pannonhalma High School} % Degree
    {High School Diploma with Honors (Ø 1.1)} % Institution
    {Vienna \& Pannonhalma (Hungary)} % Location
    {2005 - 2011 - 2013} % Date(s)
    {}
%---------------------------------------------------------
\end{cventries}

%\cvsection{Developed Software}
\begin{flushleft}
\renewcommand{\arraystretch}{0.8}
\begin{tabular}{p{3cm}p{8cm}}
\entrytitlestyle{LoanPy} & \descriptionstyle{A Python package for building models based on lexical data and simulating loanword adaptation and historical sound changes: https://github.com/LoanpyDataHub/loanpy}\\ %WHAT KIND OF MODELS
\entrytitlestyle{mutle} & \descriptionstyle{Wordle for the Mari language, playable at: \newline https://mari-wordle.anvil.app/}\\
\entrytitlestyle{copius\_api} & \descriptionstyle{A Python interface for various IPA-transcribers: https://github.com/martino-vic/copius\_api}

\end{tabular}
\end{flushleft}
\pagebreak
%\pagebreak
\cvsection{Fähigkeiten}
\begin{flushleft}
\renewcommand{\arraystretch}{0.5}
\begin{tabular}{p{1cm}p{3cm}p{2cm}p{2.5cm}p{3.3cm}p{4cm}} % Removed the extra column and adjusted widths
\multicolumn{2}{p{5cm}}{\cvsubsection{Natürliche Sprachen}} & \multicolumn{2}{p{5cm}}{\cvsubsection{Programmiersprachen}} & \multicolumn{1}{p{3.3cm}}{\cvsubsection{Tools}}& \multicolumn{1}{p{4cm}}{\cvsubsection{Soft Skills}}\\
\entrytitlestyle{Native} & \descriptionstyle{Deutsch} & \entrytitlestyle{Guru} & \descriptionstyle{Python} & \descriptionstyle{ChatGPT, Word, Excel} & \descriptionstyle{Kommunikation}\\
\entrytitlestyle{C2} & \descriptionstyle{Englisch, Ungarisch} & \entrytitlestyle{Experte} & \descriptionstyle{HTML/CSS} & \descriptionstyle{WordPress, X, fb, ig} & \descriptionstyle{Kreativität}\\
\entrytitlestyle{C1} & \descriptionstyle{Französisch, Kroatisch} & \entrytitlestyle{Mittelstufe} & \descriptionstyle{Web Development} & \descriptionstyle{Photoshop, InDesign} & \descriptionstyle{Kollaboration}\\
\entrytitlestyle{B2} & \descriptionstyle{Finnisch, Slowakisch} & \entrytitlestyle{Anfänger} & \descriptionstyle{C++, R} & \descriptionstyle{Adobe Premiere} & \descriptionstyle{Kulturkenntnis}
\end{tabular}
\end{flushleft}

%\cvsection{Förderungen}
\begin{cvhonors}

%---------------------------------------------------------
  \cvhonor
    {Finnische Nationalagentur für Bildung} % Award
    {} % Awarding Organization
    {€ 24.000} % Amount -- leave blank if not including amounts
    {2019-2020} % Date(s)
    
  \cvhonor
    {Erasmus+} % Award
    {} % Awarding Organization
    {€ 6.000} % Amount -- leave blank if not including amounts
    {2015-2016} % Date(s)

%  \cvhonor
 %   {Universität Wien} % Award
  %  {Leistungsstipendien} % Awarding Organization
   % {€ 2.000} % Amount -- leave blank if not including amounts
    %{2014, 2015} % Date(s)
\end{cvhonors}

%%-------------------------------------------------------------------------------
%	SECTION TITLE
%-------------------------------------------------------------------------------
\cvsection{Conferences Organised}

\begin{cvhonors}
    \cvhonor
    {Congressus XIII Internationalis Fenno-Ugristarum} % Award
    {Attracted sponsorship of €10, 000} % Awarding Organization
    {} % Amount -- leave blank if not including amounts
    {2022} % Date(s)

    \cvhonor
    {70. Studentische Tagung Sprachwissenschaft} % Award
    {Core Organiser Team} % Awarding Organization
    {} % Amount -- leave blank if not including amounts
    {2021} % Date(s)

    \cvhonor
    {Pre-Congressus XIII Internationalis Fenno-Ugristarum} % Award
    {Communication} % Awarding Organization
    {} % Amount -- leave blank if not including amounts
    {2020} % Date(s)

    \cvhonor
    {International Finno-Ugric Students' Conference XXXV} % Award
    {Logistics} % Awarding Organization
    {} % Amount -- leave blank if not including amounts
    {2019} % Date(s)

    \cvhonor
    {International Winter School of Finno-Ugric Studies (VI)} % Award
    {Event management} % Awarding Organization
    {} % Amount -- leave blank if not including amounts
    {2018} % Date(s)
\end{cvhonors}
%%-------------------------------------------------------------------------------
%	SECTION TITLE
%-------------------------------------------------------------------------------
\cvsection{Publikationen}


%-------------------------------------------------------------------------------
%	CONTENT
%-------------------------------------------------------------------------------

%---------------------------------------------------------
%\cvsubsection{Published}
%---------------------------------------------------------

\begin{cvpubs}
    \cvpub{Khuyagbaatar Batsuren, Gábor Bella, Aryaman Arora, \textbf{Viktor Martinovic}, Kyle Gorman, Zdeněk Žabokrtský, Amarsanaa Ganbold, Šárka Dohnalová, Magda Ševčíková, Kateřina Pelegrinová, Fausto Giunchiglia, Ryan Cotterell, and Ekaterina Vylomova. 2022. The SIGMORPHON 2022 Shared Task on Morpheme Segmentation. In Proceedings of the 19th SIGMORPHON Workshop on Computational Research in Phonetics, Phonology, and Morphology, pages 103–116, Seattle, Washington. Association for Computational Linguistics.}

    \cvpub{Martinović, Viktor. 2022. Converting Streitberg’s Gothic Dictionary to a CLDF wordlist on a Windows System. Computer-Assisted Language Comparison in Practice 5(2). 11–21. https://doi.org/10.17613/0DF3-GM47.}
\end{cvpubs}

% %---------------------------------------------------------
%\cvsubsection{In Prep}
% %---------------------------------------------------------

%\begin{cvpubs}
%\small \color{black}
%    \cvpub{PhD Thesis}
    
%    \cvpub{Automatic detection of phonological adjustments in Bislama’s lexicon
%           of English origin}
%\end{cvpubs}

% %---------------------------------------------------------

%\cvsection{Developed Software}
\begin{flushleft}
\renewcommand{\arraystretch}{0.8}
\begin{tabular}{p{3cm}p{8cm}}
\entrytitlestyle{LoanPy} & \descriptionstyle{A Python package for building models based on lexical data and simulating loanword adaptation and historical sound changes: https://github.com/LoanpyDataHub/loanpy}\\ %WHAT KIND OF MODELS
\entrytitlestyle{mutle} & \descriptionstyle{Wordle for the Mari language, playable at: \newline https://mari-wordle.anvil.app/}\\
\entrytitlestyle{copius\_api} & \descriptionstyle{A Python interface for various IPA-transcribers: https://github.com/martino-vic/copius\_api}

\end{tabular}
\end{flushleft}
\pagebreak


%-------------------------------------------------------------------------------
\end{document}
