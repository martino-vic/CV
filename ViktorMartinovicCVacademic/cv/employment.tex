\cvsection{Employment}
\begin{cventries}

  \cventry
    {MPI EVA, Department of Linguistic and Cultural Evolution} % Degree
    {Doctoral Researcher} % Institution
    {Leipzig} % Location
    {October 2022 - October 2023} % Date(s)
    {
      \begin{cvitems} % Description(s) bullet points
        \item{Modelling and simulating sound changes in horizontal transfers}
        \item{Inferring sound correspondence patterns in ancient and contemporary loanwords}
        \item{Computationally extracting and standardising information from dictionaries}
        %\item {Simulation modeeling, quantitative, use and dev of software, stand-alone, macro-simulation, informal model, inference}
      \end{cvitems}
    }
    
  \cventry
    {Austrian Foreign Ministry, Dpt. for Southeastern Europe} % Degree
    {Intern} % Institution
    {Vienna} % Location
    {June 2018 - December 2018} % Date(s)
    {
      \begin{cvitems} % Description(s) bullet points
        \item{Documenting and analysing events in the Balkans region}
      \end{cvitems}
    }

%  \cventry
%    {University of Vienna} % Degree
%    {Student assistant} % Institution
%    {Vienna} % Location
%    {October 2017 - January 2018} % Date(s)
%    {}
%  \cventry
 %   {Hungarian Embassy Paris} % Degree
  %  {Intern} % Institution
   % {Paris} % Location
    %{June - August 2016<} % Date(s)
    %{}  
    
\end{cventries}

%%keywords: simulation modelling. uncovering neglected histories. multilingualism. comparative analysis, interdisciplinary, NHL, reconstruct, explain, understand, causality, qualitative, quantitative and experimental methods, generative simulation models of the dynamics of historical and contemporary multilingualism, historical and present day data, informal models from published literatures in linguistics, historical inference on population dynamics; the use of in-development software designed for generative micro-simulation, stand-alone modular add-ons to these software environments specific to multi-linguistic phenomena. qualitative arguments (informal models) into modeling assumptions for use in either micro or macro-simulation projects. generative (forward) simulation and inference methods, demonstrated problem-solving ability, both historically and in contemporary contexts,